\documentclass[acmart]{apa6}

\usepackage{amssymb,amsmath}
\usepackage{ifxetex,ifluatex}
\usepackage{fixltx2e} % provides \textsubscript
\ifnum 0\ifxetex 1\fi\ifluatex 1\fi=0 % if pdftex
  \usepackage[T1]{fontenc}
  \usepackage[utf8]{inputenc}
\else % if luatex or xelatex
  \ifxetex
    \usepackage{mathspec}
    \usepackage{xltxtra,xunicode}
  \else
    \usepackage{fontspec}
  \fi
  \defaultfontfeatures{Mapping=tex-text,Scale=MatchLowercase}
  \newcommand{\euro}{€}
\fi
% use upquote if available, for straight quotes in verbatim environments
\IfFileExists{upquote.sty}{\usepackage{upquote}}{}
% use microtype if available
\IfFileExists{microtype.sty}{\usepackage{microtype}}{}

% Table formatting
\usepackage{longtable, booktabs}
\usepackage{lscape}
% \usepackage[counterclockwise]{rotating}   % Landscape page setup for large tables
\usepackage{multirow}		% Table styling
\usepackage{tabularx}		% Control Column width
\usepackage[flushleft]{threeparttable}	% Allows for three part tables with a specified notes section
\usepackage{threeparttablex}            % Lets threeparttable work with longtable

% Create new environments so endfloat can handle them
% \newenvironment{ltable}
%   {\begin{landscape}\begin{center}\begin{threeparttable}}
%   {\end{threeparttable}\end{center}\end{landscape}}

\newenvironment{lltable}
  {\begin{landscape}\begin{center}\begin{ThreePartTable}}
  {\end{ThreePartTable}\end{center}\end{landscape}}

  \usepackage{ifthen} % Only add declarations when endfloat package is loaded
  \ifthenelse{\equal{\string acmart}{\string man}}{%
   \DeclareDelayedFloatFlavor{ThreePartTable}{table} % Make endfloat play with longtable
   % \DeclareDelayedFloatFlavor{ltable}{table} % Make endfloat play with lscape
   \DeclareDelayedFloatFlavor{lltable}{table} % Make endfloat play with lscape & longtable
  }{}%



% The following enables adjusting longtable caption width to table width
% Solution found at http://golatex.de/longtable-mit-caption-so-breit-wie-die-tabelle-t15767.html
\makeatletter
\newcommand\LastLTentrywidth{1em}
\newlength\longtablewidth
\setlength{\longtablewidth}{1in}
\newcommand\getlongtablewidth{%
 \begingroup
  \ifcsname LT@\roman{LT@tables}\endcsname
  \global\longtablewidth=0pt
  \renewcommand\LT@entry[2]{\global\advance\longtablewidth by ##2\relax\gdef\LastLTentrywidth{##2}}%
  \@nameuse{LT@\roman{LT@tables}}%
  \fi
\endgroup}


\ifxetex
  \usepackage[setpagesize=false, % page size defined by xetex
              unicode=false, % unicode breaks when used with xetex
              xetex]{hyperref}
\else
  \usepackage[unicode=true]{hyperref}
\fi
\hypersetup{breaklinks=true,
            pdfauthor={},
            pdftitle={The title},
            colorlinks=true,
            citecolor=blue,
            urlcolor=blue,
            linkcolor=black,
            pdfborder={0 0 0}}
\urlstyle{same}  % don't use monospace font for urls

\setlength{\parindent}{0pt}
%\setlength{\parskip}{0pt plus 0pt minus 0pt}

\setlength{\emergencystretch}{3em}  % prevent overfull lines


% Manuscript styling
\captionsetup{font=singlespacing,justification=justified}
\usepackage{csquotes}
\usepackage{upgreek}



\usepackage{tikz} % Variable definition to generate author note

% fix for \tightlist problem in pandoc 1.14
\providecommand{\tightlist}{%
  \setlength{\itemsep}{0pt}\setlength{\parskip}{0pt}}

% Essential manuscript parts
  \title{The title}

  \shorttitle{Title}


  \author{First Author\textsuperscript{1}~\& Ernst-August Doelle\textsuperscript{1,2}}

  % \def\affdep{{"", ""}}%
  % \def\affcity{{"", ""}}%

  \affiliation{
    \vspace{0.5cm}
          \textsuperscript{1} Wilhelm-Wundt-University\\
          \textsuperscript{2} Konstanz Business School  }

  \authornote{
    Add complete departmental affiliations for each author here. Each new
    line herein must be indented, like this line.
    
    Enter author note here.
    
    Correspondence concerning this article should be addressed to First
    Author, Postal address. E-mail:
    \href{mailto:my@email.com}{\nolinkurl{my@email.com}}
  }


  \abstract{Enter abstract here. Each new line herein must be indented, like this
line.}
  \keywords{keywords \\

    \indent Word count: X
  }





\usepackage{amsthm}
\newtheorem{theorem}{Theorem}[section]
\newtheorem{lemma}{Lemma}[section]
\theoremstyle{definition}
\newtheorem{definition}{Definition}[section]
\newtheorem{corollary}{Corollary}[section]
\newtheorem{proposition}{Proposition}[section]
\theoremstyle{definition}
\newtheorem{example}{Example}[section]
\theoremstyle{definition}
\newtheorem{exercise}{Exercise}[section]
\theoremstyle{remark}
\newtheorem*{remark}{Remark}
\newtheorem*{solution}{Solution}
\begin{document}

\maketitle

\setcounter{secnumdepth}{0}



\textbf{CCS Concepts}

\begin{itemize}
\tightlist
\item
  Applied computing \textasciitilde{} Education \textasciitilde{}
  Learning management systems
\item
  Applied computing \textasciitilde{} Education \textasciitilde{}
  E-learning
\item
  Applied computing \textasciitilde{} Education \textasciitilde{}
  Computer-managed instruction
\end{itemize}

\textbf{Keywords}

\section{1. INTRODUCTION}\label{introduction}

In recent years, educational institutions have begun to collect student
data (REF). One area of interest is the delivery of fully online
instruction, which is becoming more prevalent (REF). Specifically,
online education is available for K-12 students who cannot or prefer not
to attend a brick-and-mortar school (REF). We are fortunate to have a
robust dataset which includes self-reported motivation as well as
behavioral trace data which was collected from the learning management
system. Our work examines the idea of educational success in terms of
student interactions with an online science course. In the current
study, we examine the educational experiences of students in online
science courses at a virtual middle school in order to characterize
their motivation to achieve and their tangible engagement with the
course in terms of trace measures.

One meaningful perspective from which to consider students' engagement
with online courses is related to their motivation to achieve. More
specifically, it is important to consider how and why students are
engaging with the course. To consider the psychological mechanisms
behind achievement is valuable because doing so may help to identify
meaningful points of intervention for educators.

Expectancy-value theory (EVT) is a key motivational framework that
explains the reasons that students are motivated to achieve (Eccles et
al., 1983). EVT posits that students are motivated to achieve when (1)
they perceive themselves to be capable of success (e.g.,
\enquote{expectancy}) and (2) they perceive present or future value in
the task at hand (e.g., \enquote{value}). Two types of value are utility
value, which refers to the degree to which students perceive that a
given task will be useful to them for some future goal, and interest
value, which refers to the level of interest students have in a given
task. In this study, we will consider utility value, interest value, and
expectancy for success as predictors of student achievement.

We investigated three research questions:

\begin{enumerate}
\def\labelenumi{\arabic{enumi}.}
\tightlist
\item
  Is motivation - operationalized as interest value, utility value and
  perceived competence for science - relatively more predictive of
  course grades as compared to other online indicators of engagement?
\item
  Which type of motivation (e.g., interest value, utility value, and
  perceived competence) is most predictive of achievement?
\item
  Which type of trace measures (e.g., time spent on course and those
  associated with participating on discussion boards) is most predictive
  of achievement?
\end{enumerate}

\section{2. METHOD}\label{method}

\subsection{2.1 Participants}\label{participants}

Participants were 499 students enrolled in online middle school science
courses in 2015-2016.

\subsection{2.2 Setting / Data Sources}\label{setting-data-sources}

The setting of this study was a public, provider of individual online
courses in a Midwestern state. In particular, the context was two
semesters (Fall and Spring) of offerings of five online science courses
(Anatomy \& Physiology, Forensic Science, Oceanography, Physics, and
Biology), with a total of 36 classes. Students completed a pre-course
survey about their self-reported motivation in science --- in
particular, their perceived competence, utility value, and interest. We
also kept track of the time students spent on the course (obtained from
the Learning Management System) and their final course grades as well as
their involvement in discussion forums. For the discussion board data,
we used the Linguistic Inquiry and Word Count (LIWC; Pennebaker, Boyd,
Jordan, \& Blackburn, 2015) to calculate the number of posts per student
and variable for the mean levels of students' cognitive processing,
positive affect, and social-related discourse evidenced by their posts.

\subsection{2.3 Procedure}\label{procedure}

At the beginning of the semester, students were asked to complete the
pre-course survey about their perceived competence, utility value, and
interest. At the end of the semester, the time students spent on the
course, their final course grades, and the contents of the discussion
forums were collected.

\subsection{2.4 Data analysis}\label{data-analysis}

\subsubsection{Preliminary Data
Wrangling}\label{preliminary-data-wrangling}

The random forest algorithm does not accept cases with missing data.
Thus, we deleted cases listwise if data were missing. This decision
eliminated 51 cases from our original dataset, to bring us to our final
sample size of 499 unique students.

\subsubsection{Main Modeling}\label{main-modeling}

For our analyses, we used Random Forest modeling (Breiman, 2001). Random
forest is an extension of decision tree modeling, whereby a collection
of decision trees are simultaneously \enquote{grown} and are evaluated
based on out-of-sample predictive accuracy (Breiman, 2001). Random
forest is random in two main ways: first, each tree is only allowed to
\enquote{see} and split on a limited number of predictors instead of all
the predictors in the parameter space; second, a random subsample of the
data is used to grow each individual tree, such that no individual case
is weighted too heavily in the final prediction. Whereas some machine
learning approaches (e.g., boosted trees) would utilize an iterative
model-building approach, random forest estimates all the decision trees
at once. In this way, each tree is independent of every other tree.
Thus, the random forest algorithm provides a robust regression approach
that is distinct from other modeling approaches. The final random forest
model aggregates the findings across all the separate trees in the
forest in order to offer a collection of \enquote{most important}
variables as well as a percent variance explained for the final model.

Random forest is well suited to the research questions that we had here
because it allows for nonlinear modeling. We hypothesized complex
relationships between students' motivation, their engagement with the
online courses, and their achievement. For this reason, a traditional
regressive or structural equation model would have been insufficient to
model the parameter space we were interesting in modeling.

Our random forest model had one outcome and eleven predictors. The
outcome was the final course grade that the student earned. The
predictor variables included motivation variables (interest value,
utility value, and science perceived competence) and trace variables
(the amount of time spent in the course, the course name, the number of
discussion board posts over the course of the semester, the mean level
of cognitive processing evident in discussion board posts, the positive
affect evident in discussion board posts, the negative affect evident in
discussion board posts, and the social-related discourse evident in
their discussion board posts). We used this random forest model to
address all three of our research questions.

In this study, we used the package randomForest in R (Liaw, 2018). 500
trees were grown as part of our random forest. We partitioned the data
before conducting the main analysis so that neither the training and
testing dataset would not be disproportionately representative of
high-achieving or low-achieving students. The training dataset consisted
of 80\% of the original data (n = 400 cases), whereas the testing
dataset consisted of 20\% of the original data (n = 99 cases). We built
our random forest model on the training dataset, and then evaluated the
model on the testing dataset. Three variables were tried at each node.
To interpret our findings, we examined three main things: (1) predictive
accuracy of the random forest model, (2) variable importance, and (3)
variance explained by the final random forest model.

\section{3. RESULTS}\label{results}

The predictive accuracy of our random forest model was assessed by
examining the difference between the predicted values for the testing
dataset and the actual values. We found that the absolute value of the
average difference between the predicted and actual value for final
grades was 11.8\%. This indicates

The variance explained by our random forest model was 57.21\%. Below, we
will discuss in detail the specific findings for each of our research
questions, which concern the variable importance plots. Variable
importance plots are interpreted based on the incremental percent change
in mean-squared-error (MSE) if a given variable is scrambled in the
original dataset (REF). In other words, variable importance plots help
to answer the question: if a variable is scrambled so as not to relate
to the outcome in any systematic way, how much does this randomization
affect the mean squared error? If a variable's scrambling results in a
large change in MSE, it is thought to be more important.

\subsection{Research Question 1}\label{research-question-1}

Research question 1 asked whether motivation was a better predictor of
achievement than behavioral engagement indicators. With respect to
research question 1, the variable importance plot for final grade
indicated that the change in mean squared error was more strongly
affected by trace variables than motivation measures. The most
predictive variable was the number of discussion posts, followed by the
amount of time spent in the course. The course identifier, evidence for
negative affect in the discussion posts, and level of cognitive
processing associated with the discussion posts were the predictors that
were next in terms of importance. All of these predictors were more
important than all of the motivation variables.

\subsection{Research Question 2}\label{research-question-2}

Research question 2 asked which of the motivation variables was most
predictive of course achievement. Among motivation variables, utility
value was most important, followed by perceived competence. Interest
value was the least predictive of the motivation variables; indeed,
interest value was the least predictive of all variables in the random
forest model.

\subsection{Research Question 3}\label{research-question-3}

Research question 3 asked which of the trace variables was most
predictive of course achievement. The most predictive variable in terms
of achievement was the number of posts on the discussion board. This was
the most predictive of all the variables in the model.

\section{4. DISCUSSION}\label{discussion}

Overall, our random forest model explained a large amount of the
variance in achievement in this study (e.g., 57.21\%). However, the
predictive accuracy of the model was not tuned as perfectly as it could
have been: the absolute value of the average difference between the
predicted final grade and actual final grade was 11.8\%. This
discrepancy suggests that whereas our model did a good job at explaining
variance in the outcome of achievement, it did not perform as well in
its prediction of \enquote{unseen} test data as would be ideal. Future
research should thus explore whether the predictive accuracy of the
model could be further developed. Even so, our predictive accuracy is
not so low as to be unhelpful. Rather, this study offers interesting
insights as to the relative importance of motivation constructs and
trace measures of enagement in terms of explanatory power in explaining
middle school students' online science course grades.

Surprisingly, we found that trace measures of engagement with a learning
management system were more predictive of student achievement than
motivation variables.

\subsection{Limitations}\label{limitations}

This study was limited in some important ways. First, we chose to
operationalize achievement as final course grade. Future work could
examine other meaningful outcomes. Second, we deleted 51 cases listwise
from our data. This listwise deletion of cases is potentially
problematic because students with some missing data could differ in
important ways from students with no missing data. Last, we did not
account for the number of discussion posts that were required in a given
course and it is important that future research endeavor to explore
whether this plays a role in predicting the outcome.

\subsection{Implications}\label{implications}

As more and more courses move online, data will continue to accumulate
at rapid rates. It is important that educators and administrators
consider the implications of computer-mediated instruction. This study
suggests that the measurement of students' engagement with courses is
helpful in understanding their achievement in these courses. Trace data
is valuable to collect and it could be valuable for educators to
consider it more thoroughly. This study also offers implications in
terms of the motivation constructs studied. The importance of the
engagement with the course through discussion board posts in terms of
predicting final grade suggests that perhaps it is valuable for students
to post even if they are not intrinsically motivated to do so. Future
research should explore the complex relations between student motivation
and course engagement, especially insofar as to examine characteristics
of the online experience that could make these relations different than
the patterns of relations that would be evident in a face-to-face
classroom.

\begin{center}\rule{0.5\linewidth}{\linethickness}\end{center}

We will delete the section of text immediately below this; it's just
notes from the call
*********************************************************************************************************
*Below are the specific questions from the LAK website that we should
reflect on\ldots{} maybe not just in the discussion, but also in other
parts of the work as well.

\begin{itemize}
\item
  What is the most surprising part of your results? Was this surprise
  shared by the people involved?
\item
  Can you justify why you used one specific methodology instead of an
  alternative?
\item
  What is the the value and potential impact of your initiative at
  scale?
\item
  What changes in teaching and learning activities you envision that
  could be realistically derived from your work?
\item
  What is the target audience for your study?
\end{itemize}

\newpage

\section{Works Cited}\label{works-cited}

Breiman, L. (2001). Random forests. Machine Learning, 45, 5--32.
\url{doi:10.1023/A:1010933404324}

Liaw, A. (2018). Package \enquote{randomForest}: Breiman and Cutler's
Random Forests for Classification and Regression.
\url{https://cran.r-project.org/web/packages/randomForest/randomForest.pdf}

Pennebaker, Boyd, Jordan, \& Blackburn, 2015 - LIWC package

\section{References}\label{references}

\begingroup
\setlength{\parindent}{-0.5in} \setlength{\leftskip}{0.5in}

\hypertarget{refs}{}

\endgroup






\end{document}
