\documentclass[man]{apa6}
\usepackage{lmodern}
\usepackage{amssymb,amsmath}
\usepackage{ifxetex,ifluatex}
\usepackage{fixltx2e} % provides \textsubscript
\ifnum 0\ifxetex 1\fi\ifluatex 1\fi=0 % if pdftex
  \usepackage[T1]{fontenc}
  \usepackage[utf8]{inputenc}
\else % if luatex or xelatex
  \ifxetex
    \usepackage{mathspec}
  \else
    \usepackage{fontspec}
  \fi
  \defaultfontfeatures{Ligatures=TeX,Scale=MatchLowercase}
\fi
% use upquote if available, for straight quotes in verbatim environments
\IfFileExists{upquote.sty}{\usepackage{upquote}}{}
% use microtype if available
\IfFileExists{microtype.sty}{%
\usepackage{microtype}
\UseMicrotypeSet[protrusion]{basicmath} % disable protrusion for tt fonts
}{}
\usepackage{hyperref}
\hypersetup{unicode=true,
            pdftitle={The title},
            pdfauthor={First Author~\& Ernst-August Doelle},
            pdfkeywords={keywords},
            pdfborder={0 0 0},
            breaklinks=true}
\urlstyle{same}  % don't use monospace font for urls
\usepackage{graphicx,grffile}
\makeatletter
\def\maxwidth{\ifdim\Gin@nat@width>\linewidth\linewidth\else\Gin@nat@width\fi}
\def\maxheight{\ifdim\Gin@nat@height>\textheight\textheight\else\Gin@nat@height\fi}
\makeatother
% Scale images if necessary, so that they will not overflow the page
% margins by default, and it is still possible to overwrite the defaults
% using explicit options in \includegraphics[width, height, ...]{}
\setkeys{Gin}{width=\maxwidth,height=\maxheight,keepaspectratio}
\IfFileExists{parskip.sty}{%
\usepackage{parskip}
}{% else
\setlength{\parindent}{0pt}
\setlength{\parskip}{6pt plus 2pt minus 1pt}
}
\setlength{\emergencystretch}{3em}  % prevent overfull lines
\providecommand{\tightlist}{%
  \setlength{\itemsep}{0pt}\setlength{\parskip}{0pt}}
\setcounter{secnumdepth}{0}
% Redefines (sub)paragraphs to behave more like sections
\ifx\paragraph\undefined\else
\let\oldparagraph\paragraph
\renewcommand{\paragraph}[1]{\oldparagraph{#1}\mbox{}}
\fi
\ifx\subparagraph\undefined\else
\let\oldsubparagraph\subparagraph
\renewcommand{\subparagraph}[1]{\oldsubparagraph{#1}\mbox{}}
\fi

%%% Use protect on footnotes to avoid problems with footnotes in titles
\let\rmarkdownfootnote\footnote%
\def\footnote{\protect\rmarkdownfootnote}


  \title{The title}
    \author{First Author\textsuperscript{1}~\& Ernst-August
Doelle\textsuperscript{1,2}}
    \date{}
  
\shorttitle{Title}
\affiliation{
\vspace{0.5cm}
\textsuperscript{1} Wilhelm-Wundt-University\\\textsuperscript{2} Konstanz Business School}
\keywords{keywords\newline\indent Word count: X}
\usepackage{csquotes}
\usepackage{upgreek}
\captionsetup{font=singlespacing,justification=justified}

\usepackage{longtable}
\usepackage{lscape}
\usepackage{multirow}
\usepackage{tabularx}
\usepackage[flushleft]{threeparttable}
\usepackage{threeparttablex}

\newenvironment{lltable}{\begin{landscape}\begin{center}\begin{ThreePartTable}}{\end{ThreePartTable}\end{center}\end{landscape}}

\makeatletter
\newcommand\LastLTentrywidth{1em}
\newlength\longtablewidth
\setlength{\longtablewidth}{1in}
\newcommand{\getlongtablewidth}{\begingroup \ifcsname LT@\roman{LT@tables}\endcsname \global\longtablewidth=0pt \renewcommand{\LT@entry}[2]{\global\advance\longtablewidth by ##2\relax\gdef\LastLTentrywidth{##2}}\@nameuse{LT@\roman{LT@tables}} \fi \endgroup}


\DeclareDelayedFloatFlavor{ThreePartTable}{table}
\DeclareDelayedFloatFlavor{lltable}{table}
\DeclareDelayedFloatFlavor*{longtable}{table}
\makeatletter
\renewcommand{\efloat@iwrite}[1]{\immediate\expandafter\protected@write\csname efloat@post#1\endcsname{}}
\makeatother
\usepackage{lineno}

\linenumbers

\authornote{Add complete departmental affiliations for each
author here. Each new line herein must be indented, like this line.

Enter author note here.

Correspondence concerning this article should be addressed to First
Author, Postal address. E-mail:
\href{mailto:my@email.com}{\nolinkurl{my@email.com}}}

\abstract{
One or two sentences providing a \textbf{basic introduction} to the
field, comprehensible to a scientist in any discipline.

Two to three sentences of \textbf{more detailed background},
comprehensible to scientists in related disciplines.

One sentence clearly stating the \textbf{general problem} being
addressed by this particular study.

One sentence summarizing the main result (with the words ``\textbf{here
we show}'' or their equivalent).

Two or three sentences explaining what the \textbf{main result} reveals
in direct comparison to what was thought to be the case previously, or
how the main result adds to previous knowledge.

One or two sentences to put the results into a more \textbf{general
context}.

Two or three sentences to provide a \textbf{broader perspective},
readily comprehensible to a scientist in any discipline.


}

\begin{document}
\maketitle

\section{CCS Concepts}\label{ccs-concepts}

Applied computing \textasciitilde{} Education \textasciitilde{} Learning
management systems Applied computing \textasciitilde{} Education
\textasciitilde{} E-learning Applied computing \textasciitilde{}
Education \textasciitilde{} Computer-managed instruction

\section{Keywords}\label{keywords}

\section{1. INTRODUCTION}\label{introduction}

In recent years, educational institutions have begun to collect student
data (REF). One area of interest is the delivery of fully online
instruction, which is becoming more prevalent (REF). Specifically,
online education is available for K-12 students who cannot or prefer not
to attend a brick-and-mortar school (REF). We seek to examine in the
current study the educational experiences of students in online science
courses at a virtual middle school.

One meaningful perspective from which to consider students' engagement
with online courses is related to their motivation to achieve. More
specifically, it is important to consider how and why students are
engaging with the course. To consider the psychological mechanisms
behind achievement is valuable because doing so may help to identify
meaningful points of intervention for educators.

Expectancy-value theory (EVT) is a key motivational framework that
explains the reasons that students are motivated to achieve (Eccles et
al., 1983). EVT posits that students are motivated to achieve when (1)
they perceive themselves to be capable of success (e.g.,
\enquote{expectancy}) and (2) they perceive present or future value in
the task at hand (e.g., \enquote{value}). Two types of value are utility
value, which refers to the degree to which students perceive that a
given task will be useful to them for some future goal, and interest
value, which refers to the level of interest students have in a given
task. In this study, we will consider utility value, interest value, and
expectancy for success as predictors of student achievement.

We are fortunate to have a robust dataset which includes self-reported
motivation as well as behavioral trace data which was collected from the
learning management system. (MAYBE SAY MORE ABOUT THIS IN THE METHOD
INSTEAD OF INTRO\ldots{}? - EAB 9.21.2018)

We investigated three research questions: (1) Is motivation -
operationalized as interest value, utility value and perceived
competence for science - relatively more predictive of course grades as
compared to other online indicators of engagement? (2) Which types of
motivation (e.g., interest value, utility value, and perceived
competence) is most predictive of achievement? (3) Which types of trace
measures (e.g.,\\
- cogproc - social - posemo - negemo - perscon - n (this is the number
of posts) are most predictive?

\subsection{Notes on Intro from the
call}\label{notes-on-intro-from-the-call}

We welcome theoretical, methodological, empirical and technical
contributions to all fields related to learning analytics. Related to
our special theme the following topics are of particular interest:

\begin{itemize}
\tightlist
\item
  Universal design for learning promotes an inclusive approach to the
  curriculum -- how can learning analytics support curriculum design and
  revision from this perspective?
\item
  How can analytics be applied in ways that support inclusion and
  success?
\item
  How can the training of data scientists be made more inclusive?
\item
  What does educational success look like, and how can it be supported?
\item
  How can systematic biases (e.g.~related to diversity) in our analytics
  algorithms be identified, reflected, and possibly avoided?
\end{itemize}

\section{A. BACKGROUND AND RELATED
WORK}\label{a.-background-and-related-work}

! We might not need this section, I got the idea from a full paper. I
think it overlaps with intro

\section{2. METHOD}\label{method}

\subsection{2.1 Participants}\label{participants}

Participants were \#\#\#\#\#\# students enrolled in online middle school
science courses in \_\_\_\_years\_\_\_\_.

\subsection{2.2 Setting / Data Sources}\label{setting-data-sources}

\subsection{2.3 Procedure}\label{procedure}

\subsection{2.4 Data analysis}\label{data-analysis}

We used R (Version 3.4.3; R Core Team, 2017) for all our analyses.

For our analyses, we used

\section{3. RESULTS}\label{results}

\section{4. DISCUSSION}\label{discussion}

*Below are the specific questions from the LAK website that we should
reflect on\ldots{} maybe not just in the discussion, but also in other
parts of the work as well.

\begin{itemize}
\item
  What is the most surprising part of your results? Was this surprise
  shared by the people involved?
\item
  Can you justify why you used one specific methodology instead of an
  alternative?
\item
  What is the the value and potential impact of your initiative at
  scale?
\item
  What changes in teaching and learning activities you envision that
  could be realistically derived from your work?
\item
  What is the target audience for your study?
\end{itemize}

\newpage

\section{References}\label{references}

\begingroup
\setlength{\parindent}{-0.5in} \setlength{\leftskip}{0.5in}

\hypertarget{refs}{}
\hypertarget{ref-R-base}{}
R Core Team. (2017). \emph{R: A language and environment for statistical
computing}. Vienna, Austria: R Foundation for Statistical Computing.
Retrieved from \url{https://www.R-project.org/}

\endgroup


\end{document}
